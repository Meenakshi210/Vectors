\let\negmedspace\undefined\let\negthickspace\undefined
\documentclass[journal,12pt,onecolumn]{IEEEtran}       \def\inputGnumericTable{}                                 %%
\usepackage{cite}\usepackage{amsmath,amssymb,amsfonts,amsthm}
\usepackage{algorithmic}\usepackage{graphicx}
\usepackage{textcomp}\usepackage{xcolor}
\usepackage{txfonts}\usepackage{listings}
\usepackage{enumitem}\usepackage{mathtools}
\usepackage{gensymb}\usepackage[breaklinks=true]{hyperref}
\usepackage{tkz-euclide} % loads  TikZ and tkz-base\usepackage{listings}
\usepackage{xparse}
\usepackage{gvv}%
%\usepackage{setspace}%\usepackage{gensymb}
%\doublespacing%\singlespacing
%\usepackage{graphicx}
%\usepackage{amssymb}%\usepackage{relsize}
%\usepackage[cmex10]{amsmath}%\usepackage{amsthm}
%\interdisplaylinepenalty=2500%\savesymbol{iint}
%\usepackage{txfonts}%\restoresymbol{TXF}{iint}
%\usepackage{wasysym}%\usepackage{amsthm}
%\usepackage{iithtlc}%\usepackage{mathrsfs}
%\usepackage{txfonts}%\usepackage{stfloats}
%\usepackage{bm}%\usepackage{cite}
%\usepackage{cases}%\usepackage{subfig}
%\usepackage{xtab}%\usepackage{longtable}
%\usepackage{multirow}%\usepackage{algorithm}
%\usepackage{algpseudocode}%\usepackage{enumitem}
%\usepackage{mathtools}%\usepackage{tikz}
%\usepackage{circuitikz}%\usepackage{verbatim}
%\usepackage{tfrupee}%\usepackage{stmaryrd}
%\usetkzobj{all}    \usepackage{color}                                            %%
    \usepackage{array}                                            %%    \usepackage{longtable}                                        %%
    \usepackage{calc}                                             %%    \usepackage{multirow}                                         %%
    \usepackage{hhline}                                           %%    \usepackage{ifthen}                                           %%
 %optionally (for landscape tables embedded in another document): %%    \usepackage{lscape}     
%\usepackage{multicol}%\usepackage{chngcntr}
%\usepackage{enumerate}
%\usepackage{wasysym}%\documentclass[conference]{IEEEtran}
%\IEEEoverridecommandlockouts% The preceding line is only needed to identify funding in the first footnote. If that is unneeded, please comment it out.
\newtheorem{theorem}{Theorem}[section]
\newtheorem{problem}{Problem}\newtheorem{proposition}{Proposition}[section]
\newtheorem{lemma}{Lemma}[section]\newtheorem{corollary}[theorem]{Corollary}
\newtheorem{example}{Example}[section]\newtheorem{definition}[problem]{Definition}
%\newtheorem{thm}{Theorem}[section] %\newtheorem{defn}[thm]{Definition}
%\newtheorem{algorithm}{Algorithm}[section]%\newtheorem{cor}{Corollary}
\newcommand{\BEQA}{\begin{eqnarray}}\newcommand{\EEQA}{\end{eqnarray}}
\newcommand{\define}{\stackrel{\triangle}{=}}\theoremstyle{remark}
\newtheorem{rem}{Remark}
%\bibliographystyle{ieeetr}
\begin{document}

Consider a triangle with vertices
		\begin{align}
			\label{eq:tri-pts}
			\vec{A} = \myvec{1\\ 5},\,
			\vec{B} = \myvec{-1\\ 0},\,
			\vec{C} = \myvec{2\\ -3}
		\end{align}
\begin{table}[!ht]
	\input{/home/meena/vectors/tables/table1.tex}
	\caption{Triangle ABC}
	\label{table1}	
\end{table}
\begin{figure}[H]
\includegraphics[width=\columnwidth]{/home/meena/vectors/figs/fig1.png}
\caption{Triangle ABC}
\label{fig:i_triangle_py}
\end{figure}

\begin{table}[!ht]
	\input{/home/meena/vectors/tables/table2.tex}
	\caption{Medians}
	\label{table1}	
\end{table}
\begin{figure}[H]
\includegraphics[width=\columnwidth]{/home/meena/vectors/figs/fig2.png}
\caption{Medians}
\label{fig:i_triangle_py}
\end{figure}

\begin{table}[!ht]
	\input{/home/meena/vectors/tables/table3.tex}
	\caption{Altitudes}
	\label{table1}	
\end{table}
\begin{figure}[H]
\includegraphics[width=\columnwidth]{/home/meena/vectors/figs/fig3.png}
\caption{Altitudes}
\label{fig:i_triangle_py}
\end{figure}

\begin{table}[!ht]
	\input{/home/meena/vectors/tables/table4.tex}
	\caption{Perpendicual bisectors}
	\label{table1}	
\end{table}
\begin{figure}[H]
\includegraphics[width=\columnwidth]{/home/meena/vectors/figs/fig4.png}
\caption{Perpendicular bisectors}
\label{fig:i_triangle_py}
\end{figure}

\begin{table}[!ht]
	\input{/home/meena/vectors/tables/table5.tex}
	\caption{Angle bisectors}
	\label{table1}	
\end{table}
\begin{figure}[H]
\includegraphics[width=\columnwidth]{/home/meena/vectors/figs/fig5.png}
\caption{Angle bisectors}
\label{fig:i_triangle_py}
\end{figure}

\end{document}


